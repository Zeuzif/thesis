\addcontentsline{toc}{chapter}{Résumé}

\begin{abstract}[Résumé]
Le covoiturage s'impose comme élément clé dans le transport urbain durable, car il réduit le nombre de véhicules en circulation, et par conséquent la congestion, la pollution et les émissions de gaz à effet de serre.\newline
Les conducteurs peuvent prendre en charge des passagers en leurs procurant des sièges à condition qu'ils ont une direction similaire.\newline
Les méthodes traditionnelles qu'on connait aujourd'hui conviennent aux voyages interurbains (longue distance).\newline
L'objectif de notre travail est de penser et implémenter un service qui va faire jumeler automatiquement les offres et les demandes sans accord préalable entre le conducteur et le passager.\newline La partie mise en œuvre concerne un algorithme de correspondance, qui vérifie si un conducteur peut emmener un passager avec lui sans violer la contrainte du temps de détour maximal, et des applications mobiles responsables de donner un aspect visuel à ces résultats et interagir avec le client.\newline
Dans ce document je me suis penché sur la partie mobile destinée à tourner sur les machines iOS.
\end{abstract}


\begin{abstract}[Abstract]
Carpooling is a key element in sustainable urban transport,as it reduces the traffic and consequently the emission of greenhouse gases. \newline
Drivers can take passengers with them in their car if they are going in the same direction.\newline
The main goal of our work is to think and implement a service that will automatically match offers and requests without prior agreement between the driver and the passage.\newline
The implementation part concerns a matching algorithm that checks whether a driver can take a passenger with him without violating the constraint of the maximum detour time, and mobile applications responsible for giving a visual aspect to these results and interacting with the customer.\newline
In this document I have focused on the mobile part intended to run on iOS machines.
\end{abstract}


\begin{abstract}[\RL{ملخص}]
\RL{
يعد استخدام السيارات بالمشاركة عنصرًا رئيسيًا في النقل الحضري المستدام ، حيث يقلل من حركة المرور وبالتالي انبعاثات غازات الدفيئة.
يمكن للسائقين اصطحاب الركاب معهم في سيارتهم إذا كانوا يسيرون في نفس الاتجاه.
الهدف الرئيسي من عملنا هو التفكير في وتنفيذ خدمة تتطابق تلقائيًا مع العروض والطلبات دون موافقة مسبقة بين السائق و المسافر. يتعلق جزء التنفيذ بخوارزمية مطابقة تتحقق مما إذا كان بإمكان السائق اصطحاب راكب معه دون انتهاك قيود الحد الأقصى لوقت الالتفاف ، وتطبيقات الهاتف المحمول المسؤولة عن إعطاء جانب مرئي لهذه النتائج والتفاعل مع العميل.
لقد ركزت في هذا المستند على الجزء المحمول المقصود تشغيله على أجهزة \LR{iOS}.
}
\end{abstract}

