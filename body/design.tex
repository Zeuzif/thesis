\chapter{Analyse des besoins}
Ce chapitre rassemble les ressources pour mener l'expression des besoins. Nous allons commencer par l'identification des acteurs, puis nous passons à la spécification des besoins et faire une analyse en s'appuyant sur UML, enfin nous donnons un aperçu sur l'algorithme de matching qui est le cœur de notre métier.
% Containing a comprehensive description of the design chosen, how it addresses the problem, and why it is designed the way it is.
\section{Identification des acteurs}
Dans cette section on va décrire \textbf{QUI} utilisera notre système. Nous commençons dans un premier lieu par considérer notre système comme une boite noire pour répondre à la question qui va interagir avec notre logiciel ?\newline
les entités ou acteurs qui entrent en jeu sont :
\begin{itemize}
	\item[$\bullet$] Utilisateur : Après son inscription, il peut consulter les offres de covoiturage planifiées, mais aucun autre service ne sera pour lui opérationnel tant qu'il n'a pas terminé le processus KYC \nomenclature{KYC}{Know Your Customer.}(Know Your Customer), on détaille par la suite ce que le KYC dans notre cas.
	\item[$\bullet$] Chauffeur: doit pourvoir ajouter des offres de covoiturage planifiées ou instantanées.
	\item[$\bullet$] Passager: doit pouvoir faire des demandes de covoiturage instantanées et de réserver sa place dans les autres planifiées.
\end{itemize}
Il convient de noter que nos acteurs ne sont pas seuelement des personnes physiques, puisque celle-ci peut à la fois être un conducteur et un passager.\newline
La figure \Figref{contextDiagram.png} donne une représentation graphique du diagramme du contexte statique sans rentrer dans le détail des cardinalités.
\fig{contextDiagram.png}{Diagramme de contexte statique}{0.7}

\section{Spécification des besoins}
\subsection{Besoins fonctionnels} % (fold)
\label{sub:besoins_fonctionnels}
Après la définition des acteurs de notre application, on détermine les besoins de chaque acteur pour répondre à la question \textbf{QUI} devra pouvoir faire \textbf{QUOI} ?\newline
L'expression de la liste des fonctionnalités est comme suit :
\begin{itemize}
	\item[$\bullet$] \textbf{Inscription} : chaque utilisateur doit pouvoir s'inscrire et activer son compte en vérifiant son numéro du téléphone et son émail.
	\item[$\bullet$] \textbf{Consultation des offres} : chaque utilisateur doit pouvoir après activation du compte consulter la liste des offres planifiées.
	\item[$\bullet$] \textbf{Terminer le KYC} : l'utilisateur doit pouvoir prouver son identité afin que l’ensemble des services soient accessibles (Identification doit être automatisée).
\end{itemize}
Tout service de ceux qui suivent n'est exploitable que si l'utilisateur a terminé le processus KYC.
\begin{itemize}
	\item[$\bullet$] \textbf{Ajout des offres} : un chauffeur peut publier des offres de covoiturages planifiées.
	\item[$\bullet$] \textbf{lancement du trajet} : un conducteur peut lancer un covoiturage instantané, l’itinéraire optimal doit s’afficher tout au long de la course. Il reçoit aussi des notifications de prise en charge des passagers.
	\item[$\bullet$] \textbf{Demande de covoiturage} : un passager peut demander un covoiturage instantané (la saisie de la demande par le système doit être en temps réel), ou consulter la liste des covoiturages planifiés afin de réserver sa place.
	\item[$\bullet$] \textbf{Réseau des utilisateurs} : Les utilisateurs peuvent avoir un réseau d'amis, ainsi que, d'importer les contacts ou d'inviter leurs camarades dans les réseaux sociaux.
	\item[$\bullet$] \textbf{Le matching} : l'affectation des conducteurs aux passagers offre un modèle riche qui tient compte non seulement de l'adéquation des itinéraires, mais aussi de l'âge, du sexe et le nombre de valises.
	\item[$\bullet$] \textbf{Historique des trajets} : tout utilisateur doit avoir la liste de ses trajets futurs ou ceux qui ont été faits.
	\item[$\bullet$] \textbf{Historique des places favorites} : tout utilisateur peut garder une liste de place favorites pour faciliter la recherche des positions.
	\item[$\bullet$] \textbf{Annulation} : tout utilisateur doit pouvoir annuler des voyages.
	\item[$\bullet$] \textbf{Paiement} : Un passager doit pouvoir payer soit par cash ou par la création d'un portefeuille (Wallet) qui lui permet d'envoyer, recevoir, contrôler ses transactions.
\end{itemize}
Cette liste n'est pas exhaustive, et certains processus métiers ont été éliminés pour des raisons du degré de leurs importances et d’allégement de ce document.
% subsection besoins_fonctionnels (end)

\subsection{Besoins non fonctionnels} % (fold)
\label{sub:besoins_non_fonctionnels}
Outre les besoins fonctionnels, un système d'informations doit répondre à un ensemble d'exigences non fonctionnelles.
\begin{itemize}
	\itemb \textbf{Performance} : le chargement rapide de l’application, ouverture d’écran, des délais de rafraîchissement et le temps de réponse de l'algorithme de matching.
	\itemb \textbf{Disponibilité} : l'application doit être opérationnelle à n'importe quel moment.
	\itemb \textbf{Architecture} : le respect du modèle du développement logiciel imposer par Apple.
	\itemb \textbf{Tests} : les tests unitaires doivent être écrits et couvrir 90\% du code.
	\itemb \textbf{Ergonomie} : la densité d’éléments sur les écrans, la disposition et le flux, les couleurs.
	\itemb \textbf{Interface graphique} : les interfaces graphiques doivent être responsive c'est à dire, elles s'adaptent aux différentes tailles d’écran des appareils iOS. (Dans notre cas on va développer notre application que pour le mode portrait)
	\itemb \textbf{Sécurité} : les différents échanges entre les composants de l'application doivent être chiffrés. l’accès à l'API se limite à nos applications clientes et on doit protéger les données de l'utilisateur exemple : infos personnelles.
	\itemb \textbf{Extensibilité} : le projet se développe dans un petit périmètre au sein d'une start-up qui cherche encore son business plan, ce qui implique l'ajout dans l'avenir de plusieurs autres fonctionnalités.
\end{itemize}
En plus de ces besoins l'application a des défis à surmonter vu que nous développons dans un environnement mobile : 
\begin{itemize}
	\item les petites tailles d’écrans.
	\item la vie de la batterie.
	\item performances du matériel.
\end{itemize}
% subsection besoins_non_fonctionnels (end)

\section{Modélisation métier}
Nous avons identifié les acteurs et un certain nombre de fonctionnalités. Maintenant il est temps de définir en détail chaque fonction et l'attribuer à un acteur.
Nous allons tracer le diagramme de cas d'utilisation, quelques diagrammes de séquence système et un diagramme de classe d'analyse.\newline
Vu la méthodologie que nous avons adoptée, nous n’avons pas fait une analyse UML détaillée des besoins, ce qui veut dire que nous sommes limités aux diagrammes métier sans faire les diagrammes de conception puisque nous étions en communication permanente avec le client (product owner).
\subsection{Diagramme de cas d'utilisation} % (fold)
\label{sub:diagrammes_de_cas_d_utilisation}
Un cas d’utilisation (use case) représente un lot d’actions qui sont réalisées par le système et qui met donc en évidence de quelle façon les acteurs utiliseront le logiciel.

\fig{usecasediagram.png}{Diagramme de cas d'utilisation}{0.7}
% subsection diagrammes_de_cas_d_utilisation (end)

\subsection{Diagrammes de séquences} % (fold)
\label{sub:diagrammes_de_séquence}
Afin de faire une validation des cas d'utilisation, et compléter le diagramme des use case en mettant en
évidence les acteurs et leurs interactions avec le système d’un point de vue temporel, les digrammes de séquence nous sert de support.\newline
Chaque sous-section ci-dessous contient une fiche descriptive et le diagramme de séquence correspondant.
\subsubsection{Inscription} % (fold)
\begin{table}[H]
\begin{center}
    \begin{tabular}{ | l | p{10cm} |}
    \hline
    Titre & Inscription \\ \hline
    Acteur(s) & User \\ \hline
    Description succincte & l'inscription doit être possible pour tout utilisateur qui a déjà installé l'application \\ \hline
    Pré-conditions & aucunes \\ \hline
    Démarrage & l'utilisateur appuie sur le bouton inscription \\ \hline
    Post-conditions & création d'un nouveau utilisateur \\ \hline
    \end{tabular}
    \caption{Description textuelle du use case inscription}
\end{center}
\end{table}
\fig{sequence_insc.png}{Diagramme de séquence inscription}{0.4}
% subsubsection inscription (end)

\subsubsection{Terminer KYC} % (fold)
\begin{table}[H]
\begin{center}
    \begin{tabular}{ | l | p{10cm} |}
    \hline
    Titre & Terminer KYC \\ \hline
    Acteur(s) & User \\ \hline
    Description succincte & après son inscription un utilisateur doit valider son identité pour pouvoir profiter de tous les services \\ \hline
    Pré-conditions & inscription \\ \hline
    Démarrage & l'utilisateur clique sur le popup qui le notifie de vérifier son compte \\ \hline
    Post-conditions & vérification de l'utilisateur \\ \hline
    \end{tabular}
    \caption{Description textuelle du use case terminer le KYC}
\end{center}
\end{table}
\fig{sequence_kyc}{Diagramme de séquence de validation du processus KYC}{0.9}
% subsubsection processus_kyc (end)

\subsubsection{Démarrer un trajet instantané} % (fold)
\begin{table}[H]
\begin{center}
    \begin{tabular}{ | l | p{10cm} |}
    \hline
    Titre & Démarrer un trajet instantané \\ \hline
    Acteur(s) & Conducteur \\ \hline
    Description succincte & un chauffeur doit pouvoir lancer une offre de covoiturage \\ \hline
    Pré-conditions & terminer le KYC \\ \hline
    Démarrage & le conducteur remplit sa position d'arrivée \\ \hline
    Post-conditions & ajout de l'offre \\ \hline
    \end{tabular}
    \caption{Description textuelle du use case démarrer un trajet instantané}
\end{center}
\end{table}
\fig{sequence_conducteur.png}{Diagramme de séquence démarrer un trajet instantané}{0.9}
% subsubsection démarrer_un_covoiturage_instantané (end)

\subsubsection{Demander un covoiturage instantané} % (fold)
\begin{table}[H]
\begin{center}
    \begin{tabular}{ | l | p{10cm} | }
    \hline
    Titre & Demander un covoiturage instantané \\ \hline
    Acteur(s) & Passager \\ \hline
    Description succincte & un chauffeur doit pouvoir lancer une demande de covoiturage \\ \hline
    Pré-conditions & terminer le KYC \\ \hline
    Démarrage & le passager rempli sa position d'arrivée \\ \hline
    Post-conditions & lancement de l'algorithme de matching \\ \hline
    \end{tabular}
    \caption{Description textuelle du use case demander un covoiturage instantané}
\end{center}
\end{table}
\fig{sequence_passager.png}{Diagramme de séquence demander un covoiturage instantané}{0.9}

\begin{tabular}{c}

\end{tabular}
% subsubsection demander_un_covoiturage_instantané (end)
% subsection diagrammes_de_séquence (end)

\subsection{Conception graphique ou IHM} % (fold)
\label{sub:ihm}
À base de l'analyse métier faite, nous avons essayé d’édifier des prototypes d'interfaces graphiques qui nous servirons de guide lors de la phase du développement.\newline
Les maquettes ont été réalisées en Adobe XD qui permet aux designers de modifier et partager facilement des prototypes interactifs avec collaborateurs et réviseurs et qui est disponible sur l'ensemble des appareils et plates-formes, dont Windows, Mac, iOS et Android.
\fig{templates.png}{Prototypes d'interfaces graphiques}{0.3}
% subsection ihm (end)

\subsection{Diagramme de classe d'analyse} % (fold)
Pour identifier les éléments du domaine, les relations et interactions entre ces éléments, le diagramme de classe fournit une représentation des entités du système qui vont interagir pour réaliser les cas d'utilisation.\newline
Les classe Offre et Demande présentes sur le diagramme dépendent du covoiturage instantané. Et la classe Preference que je n'ai pas mentionné sur le diagramme représente les préférences soit du passager soit du conducteur dans les deux cas du covoiturage (planifié, instantané) et qui a comme attributs fumeur, sexe....\newline
Les documents cités dans cette partie d'analyse sont le produit d'une première réflexion et ils ont été améliorés dans la phase du développement, donc ils ne vont pas servir comme moyens de documentation.
\fig{classe.png}{Diagramme de classe métier simplifié}{0.8}
% subsection diagramme_de_classe (end)

\section{Aperçu de l'algorithme de covoiturage} % (fold)
\label{sec:Aperçu de l'algorithme de covoiturage}
\subsection{Étapes de l'algorithme} % (fold)
Après la description formelle de nos besoins, on va décrire comment se déroule le processus de matching ou de jumelage qui est le cœur de notre métier.
\fig{logigramme.png}{Logigramme du processus de matching}{0.7}\newline
La figure \Figref{logigramme.png} décrit étape par étape les différentes tâches. Notant que ce processus se répète trois fois, si aucune offre n'est trouvée on notifie le passager, dans ce cas il a la possibilité de relancer la recherche manuellement.
% subsection subsection_name (end)

\subsection{Processus du matching} % (fold)
Pour des raisons de confidentialité, on ne va pas rentrer dans les détails d'implémentation et on va se limiter aux cas simples.\cite{schreieck2016matching}\newline
Au début du processus, un conducteur qui veut partager sa voiture avec des passagers potentiels enregistre 
une offre qu'on va noter $O = (O_D, O_A)$, où $O_D$ est le point (position géographique) de départ, et $O_A$
le point d'arrivée. En plus le conducteur nous renseigne sa position en temps réel $O_t$, qu'elle met à jour chaque fois qu'il s'est déplacé d'une distance $d$. Le système procède au calcul et au découpage du trajet O en points distancés de $d$ ce qui nous donne que $O = (O_{1}, O_{2},\ldots , O_{n})$ avec $O_{1} = O_D$ et $O_{n} = O_A$\newline
Un passager qui demande un covoiturage enregistre sa demande qu'on note $D = (D_D, D_A)$, où $D_D$ est le point (position géographique) de départ, et $D_A$ le point d'arrivée.\newline
Dans un premier temps on va considérer le scenario où on a une seule offre et une seule demande. Notons $C_D$ et $C_A$ l'ensemble des points des deux cercles ayant pour centre respectivement $D_D$ et $D_A$ et pour rayon $r$.
Pour qu'un covoiturage ait lieu il faut : 
\begin{equation}
O_t \subset C_D 
\label{eqution1}
\end{equation}
ce qui veut dire que la position actuelle du chauffeur est d'une distance moins de $r$ du passager.\newline
Deuxièmement, on va vérifier que le point d'arrivée du passager est sur le chemin du conducteur. Ce qui peut se traduire en :
\begin{equation}
(O_{1}, O_{2},\ldots , O_{n}) \cap C_D \ne \emptyset
\label{eqution2}
\end{equation}
Dans le cas de plusieurs offres l'algorithme choisit le chauffeur en calculant un score qui se détermine par la convenance des préférences du chauffeur et le passager (fumeur, bagages, sexe...). Et par le temps de détour des deux offres, par détour on apprend le temps qui va mettre le chauffeur pour prendre en charge le passager, calculé par des services de cartographie tierces.
\fig{matching.png}{Processus de matching dans le cas d'une seule offre}{0.7}\newline
% subsection algorithme (end)
% section conception (end)
\paragraph{Conclusion.} Ce deuxième chapitre avait pour finalité l'identification et analyse des besoins, et la présentation de l'algorithme passons maintenant à l'implémentation.