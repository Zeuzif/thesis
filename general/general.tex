\addcontentsline{toc}{chapter}{Introduction générale}

\begin{general}

\hspace{1cm}Suite à la révolution industrielle (milieu du 19\ieme\ siècle), les agglomérations n’ont cessé de s’agrandir et de s’étendre. Donc il va falloir repenser nos modes de transport.\newline
Avec l’avènement de la voiture, chaque individu peut se procurer le plaisir de se déplacer confortablement entre son milieu de travail et son domicile.\newline
Résultat, nos villes s’engorgent, il y a de plus en plus de voitures, de bouchons, de consommation d’énergie et de pollution.\newline
En guise de solution, les services à mobilité partagée ou covoiturage qui s’imposent comme complémentaires et rivaux aux transports en commun offrent :
\begin{itemize}
\itemb Un mode de transport écologique et économique.
\itemb Une réduction de la moyenne quotidienne du temps de déplacement.
\itemb Un aspect social et convivial à nos déplacements.
\end{itemize}
Sous cet angle de vision s’inscrit ce PFE au sein de la société Peoplin et qui consiste à modéliser et réaliser une application de covoiturage.\newline
Ce rapport synthétise le travail effectué sous forme de quatre chapitres principaux : dans la première partie nous allons définir notre projet, ensuite nous allons faire une analyse des besoins suivie par une partie d'implémentation, enfin, la dernière partie va porter sur les tests et l'évaluation de notre application.
\end{general}
