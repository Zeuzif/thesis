\addcontentsline{toc}{chapter}{Introduction générale}

\begin{general}

\hspace{1cm}Depuis le milieu du 19\ieme\ siècle, les agglomérations n’ont cessé de s’agrandir et de s’étendre. C’est la révolution industrielle. Donc il va falloir repenser nos modes de transport.\newline
Avec l’avènement de la voiture, chaque individu peut se procurer le plaisir de se déplacer confortablement entre son milieu de travail et son domicile.\newline
Résultat, Nos villes s’engorgent, il y a de plus en plus de voitures, de bouchons, de consommation d’énergie et de pollution.\newline
Solution, Les services à mobilité partagée ou covoiturage qui s’imposent comme complémentaires et rivaux aux transports en commun offrent :
\begin{itemize}
\item Un mode de transport écologique et économique.
\item Une réduction de la moyenne quotidienne du temps de déplacement.
\item Un aspect social et convivial a nos déplacements.
\end{itemize}
Sous cet angle de vision futuriste s’inscrit ce PFE au sein de la société Peoplin et qui consiste à modéliser et réaliser une application de covoiturage.\newline
Conséquence, ce document synthétise le travail effectué.
\end{general}
