\chapter{Définition du projet}

Comme tout projet a un cycle de vie qui débutera par une phase de lancement qui peut avoir une signification différente pour chaque organisation mais qui généralement clarifie trois point essentiels:
\begin{itemize}
	\item la problématique.
	\item les objectifs.
	\item les contraintes.
\end{itemize}
Nous allons dans ce chapitre présenter l'organisme d’accueil afin d'avoir une idée sur les parties prenantes, par la suite on va définir notre problématique, après on fixe nos objectifs, ensuite on fait une étude de l'existant, enfin on explique La méthodologie de démarche suivie du PFE.


\section{Présentation de l’organisme}

Dans cette section, nous présentons l’organisme d’accueil de ce projet qui est la société Peoplin.\newline
Peoplin est une start-up technologique, créée en 2018 sous la dénomination CORE TECHS, et ayant pour but de fournir des services financiers digitalisés, innovants et à forte valeur ajoutée. Un prototype de Workflow décisionnel pour l’octroi de financements entièrement digitalisé et destiné aux nouvelles banques participatives a été élaboré, cependant, le marché n’était pas encore suffisamment mature pour absorber une telle solution, ce qui a amené à la refonte de l’identité de la start-up, pour aborder un nouveau créneau, celui de la ville connectée.\newline 
Ainsi, depuis début 2019, la start-up se lancera dans la réalisation d’une solution de covoiturage dynamique, première brique de sa feuille de route à moyen et long terme, en sous-traitance à une entreprise Indienne, cette dernière effectuera le prototypage du socle global de la solution et livrera une première partie des développements demandés, la deuxième partie des développements va être poursuivie au Maroc à partir de Février 2020.\newline
Les parties prenantes impliquées dans ce projet et qui suivent son cycle de vie sont: 
\begin{itemize}
	\item Le product owner : L’entreprise Peoplin.
	\item Scrum master: Mohammed salim el azzouzi.
\end{itemize}
L'equipe du développement comporte quatre personnes:
\begin{itemize}
	\item Responsable de la partie mobile IOS.
	\item Responsable de la partie mobile Android.
	\item Deux responsables du back-end.
\end{itemize}

\fig{organigramme.jpeg}{organigramme de l’équipe de projet}{0.5}

\section{Contexte}

Ces deux dernières décennies marquent au Maroc une forte augmentation de demandes du déplacement, ce qui poussent les gouvernements a constamment amélioré les moyens de transports pour répondre à ce besoin.\newline
Mais, malgré les efforts ces moyens de transport n'arrivent la demande toujours en croissance.\newline
Résultat, la voiture individuelle reste le moyen le plus pratique et confortable pour se déplacer soit dans les villes, soit dans les zones non urbaines. Néanmoins, la mobilité humaine accrue combinée avec la forte utilisation de la voiture engendrent des effets néfastes sur la qualité de voyage et soulève des questions environnementales.\newline
En effet, l'usage abusif de la voiture est source de stresse, de charges personnelles de plus en plus en hausse (à cause du prix du carburant, de l'entretient etc...) et des gazes à effet de serre.\newline
Une solution qui s'impose comme immédiate est le partage de la voiture personnelle et qui permet à un chauffeur de la partager avec des passagers potentiels, afin d'effectuer le trajet ensemble.\newline
Donc, notre solution consiste a automatisé cette opération, en optimisant les temps de détour du chauffeur tout en laissant un maximum de chance aux passagers de trouver un conducteur qui va leurs amenés vers leurs destinations tout en nouant les liens sociaux qui sont en déclin.

\section{Problématique}

Dans le but de procurer un service de covoiturage répondant aux besoins de déplacement et palier aux problèmes écologiques, sociaux et économiques cités auparavant.\newline
Les solutions qui sont présentes sur le marché ou qui font l'objet de recherche des industriels offrent des services assez limités qui se résument dans des trajets planifiés ou des trajets entre les grandes villes.\newline
En effet, un utilisateur doit chercher une offre déjà planifié et envoyer une demande au publicateur,et récupérer ainsi les coordonnées des covoitureurs pour une éventuelle prise de contact pour un trajet bien déterminé auparavant, ce qui n'automatise pas l’opération.\newline
Dans ce cadre, en se basant sur les attentes des utilisateurs, nous nous intéressons à trouver une solution informatique qui répond aux problèmes mentionnés avant et qui ne sont pas la priorité des créateurs de solutions existantes, et qui peuvent se résumer comme suit:
\begin{itemize}
 
\item[$\bullet$] dépenses du déplacement en augmentation.
\item[$\bullet$] trajets quotidiens ennuyeux et liens sociaux en déclin.
\item[$\bullet$] l'embouteillage et la pollution.
\item[$\bullet$] sécurité soit des passagers soit des conducteurs.
 
\end{itemize}

\section{Objectifs}

Dans l'optique d'apporter de nouvelles fonctionnalités et remédier aux insuffisances et limites explicitée auparavant. Notre application permettra au covoitureurs de faire des économies, aux passagers de se déplacer conformément a leurs préférences et en toute sécurité et à réduire l’émission des gazes polluants.\newline
Effectivement, l'application permet à un chauffeur de faire un choix entre des offres instantanées (pratiques dans les zones urbains dépassant pas 30Km) et des offres entre les villes. En plus toute personne ayant l'age légal, un permis peut diviser les charges du déplacement et de l'entretient de sa voiture avec d'autres potentiels passagers. Et à un passager de se déplacer avec un prix moins cher que les transports en commun sans perdre son temps et en choisissant ses préférences (fumeur, bagages, sexe du conducteur ...).\newline
L'application proposes les chemins les plus optimales avec des temps de détour pour prendre un passager courts et rapides. dans le but de réduire la consommation d’énergie et par conséquence, réduire l’émission des gazes polluants.
En gros, l'application devrait respecter les exigences listées ci-dessous
\begin{itemize}
	\item Vérification de la carte d'identité(CIN) pour chaque utilisateur.
	\item Publier une offre de covoiturage planifiées ou lancer une course instantanée.
	\item Afficher les trajets optimaux.
	\item Effectuer des transactions monétaires.
\end{itemize}

\section{Étude de l'existant}

\section{La méthodologie de démarche suivie du PFE}

