\begin{appendices}

% e.g., User Manuals, supporting evidence for claims made in the main part of the dissertation (e.g. a copy of a user evaluation questionnaire), samples of test data, etc. Note that Appendices are optional.

\chapter{Résultats de l'étude de l'existant}
\label{annex:A}
Le Benchmarking est une technique marketing ou de gestion de la qualité qui consiste à étudier et analyser les techniques de gestion, les modes d'organisation des autres entreprises afin de s'en inspirer et d'en tirer le meilleur.\newline
Pour cela un test d’applications de covoiturage existante a été fait, ce test a pour objectifs de comprendre le déroulement d’un covoiturage digital et s’en inspirer pour le développement de l’application, évaluer la qualité de ses applications et la prendre en compte pour le projet.\newline
Pour ce projet plusieurs applications ont été testées et plusieurs problèmes ont été relevés comme décrit dans le tableau suivant :

\begin{center}
\begin{longtable}{p{1cm}|p{3cm}|p{6cm}|p{4cm}|}
\caption{Applications testées.}\\

\hline \multicolumn{1}{|c|}{\textbf{Application}} & \multicolumn{1}{c|}{\textbf{Description}} & \multicolumn{1}{c|}{\textbf{Fonctionnalités}} & \multicolumn{1}{c|}{\textbf{Problèmes}}\\ \hline 
\endfirsthead

\multicolumn{4}{c}%
{{\bfseries \tablename\ \thetable{} -- suite de la page précédente}} \\
\hline \multicolumn{1}{|c|}{\textbf{Application}} & \multicolumn{1}{|c|}{\textbf{Description}} & \multicolumn{1}{c|}{\textbf{Fonctionnalités}} & \multicolumn{1}{c|}{\textbf{Problèmes}} \\ \hline 
\endhead

\hline \multicolumn{4}{|r|}{{Suite à la page suivante}} \\ \hline
\endfoot

\hline \hline
\endlastfoot

Careem & Application pour réserver des courses sûres, abordables et fiables & Réserver une course maintenant et aller au travail, voir des ami(e)s ou faire les courses, mais aussi programmer une course plus tard idéale pour être à l’heure à l’aéroport. Avoir une estimation du prix de la course *suivre votre chauffeur. service GPS tracking est au point, ainsi nos captains peuvent facilement vous déposer à l’endroit que vous souhaitez, sans perdre 30min à vous trouver.
*Paiement en cash, en crédit Careem ou par carte bancaire.
*Service client est à la disposition des utilisateurs. & *Manque la réservation de véhicule pour le lendemain par exemple *Manque réservation de la course retour à l'aller ou bien plusieurs arrêts *Manque type de véhicule et homme ou femme *On ne peut pas avoir le prix de la course avant la réservation.\\
Roby Taxi & Application de commande de taxi. & *Roby vous informe du nombre de taxis disponibles autour de vous et le type de taxi, petit ou grand taxi.
*Indiquez votre destination en cliquant sur - « Où allez-vous ? » ou sur la montre pour une réservation en différé.
*Roby calcule le prix de la course avant de la lancer (plus de discussions sur le tarif avec le chauffeur).
*Vous êtes mis en relation avec le chauffeur qui vient vous chercher, vous avez également la possibilité de le contacter à tout moment.
*Vous payez la course sans commission, directement à votre chauffeur. Roby est un service gratuit.
*À la fin de votre trajet, vous obtenez le récapitulatif de votre course et pouvez noter votre taxi. Ainsi, vous nous aidez à améliorer notre service et qualifier nos chauffeurs.  & *difficulté d’inscription
*prix élevés des courses.
*Connexion via réseaux sociaux tels que Facebook, Google ou autre est impossible
*Non possibilité d’annulation de course.
*Bugs non fixés\\
Pip pip yalah & La plus grande communauté de covoiturage au Maroc. Transformez-vous en copilote et partez au plus près de votre adresse de départ. Et si c’est vous qui conduisez, préparez-vous à réaliser de sacrées économies et rencontres. & -Conducteur: 
• Publiez votre trajet en quelques secondes
• Choisissez qui vous emmenez avec vous
• Économisez sur les coûts du trajet
-Passager :
• Cherchez parmi les trajets rejoignant votre destination
• Comparez les points de départ, l'un de ces trajets pourrait partir du coin de la rue
• Réservez ou faites une demande de réservation
• Arrivez au plus près de votre destination finale & *Interface à améliorer.
*Lorsqu'on poste une offre à deux places par exemple l'application offre la possibilité d'accepter deux demandes uniquement et si l'un de ces deux membres s'abstient l'offre voue à  l'échec.
*Le passager n’a pas droit d’annuler un trajet.
*manque une petite fenêtre de chat pour faciliter le contact et trancher immédiatement au covoiturage.

\end{longtable}
\end{center}

\end{appendices}