\chapter{Conclusion générale et réflexions}
\label{Conclusion}
Au bout du compte, nous avons fait le tour de toutes les étapes nécessaires pour réaliser un projet que ce soit informatique ou autre.\newline
En gros, un cadrage en termes de portée, de méthodologie suivie et d'outils de collaboration a été réalisé.\newline
Ensuite nous avons percé pour faire sortir et recueillir des exigences et besoins auprès des parties prenantes.\newline 
Après une étape d'analyse, qui est en fait cruciale, nous avons approprié un langage commun adapté à notre domaine afin d’éclaircir le chemin à suivre et les processus à implémenter.\newline
Enfin la réalisation avait pour but de transformer tous ces informations et données en des livrables prêts pour l'exploitation par les clients finaux.\newline
En gros, ce projet m'a permis d'appliquer mes connaissances théoriques en faisant le suivi depuis l'expression des besoins du client, la conception, prise en charge des développements et les tests.\\[1in]
Ce que j'ai retenu de ce travail, c'est que les économies collaboratives qui décrivent les activités du domaine économique fondées sur les échanges et la collaboration entre les individus vont faire face tôt ou tard à la société de consommation qui est caractérisée par le gaspillage et la sous-utilisation des biens. Ce qui m'a poussé tout au long du projet d'être contre l'idée que le conducteur peut faire des bénéfices, ce qui fait sortir la notion de covoiturage de son contexte, mais ceci ne représente aucun souci pour les compagnies qui ne cherchent qu'à faire plus de bénéfices.
